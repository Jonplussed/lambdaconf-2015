\documentclass[pdf]{beamer}
\usepackage{amsmath}
\usepackage{graphicx}
\usepackage{hyperref}
\usepackage{listings}
\usepackage{tcolorbox}
\usepackage[all]{xy}

\mode<presentation>{}

% ------------------------------------------------------------------------------
% Theme

\usetheme[usetitleprogressbar, nosmallcapitals, protectframetitle, usetotalslideindicator]{m}

\lstloadlanguages{Haskell}
\lstnewenvironment{code}
    {\lstset{}%
      \csname lst@SetFirstLabel\endcsname}
    {\csname lst@SaveFirstLabel\endcsname}
    \lstset{
      basicstyle=\ttfamily\footnotesize,
      flexiblecolumns=false,
      basewidth={0.5em,0.45em},
      literate={+}{{$+$}}1 {/}{{$/$}}1 {*}{{$*$}}1
               {\\\\}{{\char`\\\char`\\}}1
               {=>}{{$\Rightarrow$}}2
               {forall}{{$\forall$}}2
               {->}{{$\rightarrow$}}2
               {<-}{{$\leftarrow$}}2
               {>>}{{>>}}3 {>>=}{{>>=}}3,
      commentstyle={\ttfamily\color{gray}},
      language=haskell
    }
    
% ------------------------------------------------------------------------------
% Presentation

\title{Finally \texttt{mtl}!}
\date{\today}
\author{Joseph Tel Abrahamson / @sdbo / \texttt{github.com/tel} }

\renewcommand{\to}{\ensuremath{\rightarrow}}
\DeclareMathOperator{\ty}{\texttt{ :: }}

\begin{document}

\maketitle

\begin{frame}
  \frametitle{Synopsis}
  \begin{itemize}
  \item What is ``Finally Tagless''? What's important about it?
  \item What is the ``Monad Transformer Library'', \texttt{mtl}?
  \item \texttt{mtl} is a ``finally tagless'' effect library---and that's quite
    nice!
  \end{itemize}
\end{frame}

\section{A tale of three DSLs}

\begin{frame}[fragile]
  \frametitle{Hutton's Razor}
\begin{lstlisting}
data AddLang
  = AddLangIntLit Integer
  | Add AddLang AddLang
  deriving ( Show, Eq )

interpAddLang :: AddLang -> Integer
interpAddLang = \case
  AddLangIntLit i -> i
  Add l r  -> interpAddLang l + interpAddLang r

addLangExp :: AddLang
addLangExp = Add (AddLangIntLit 1) (AddLangIntLit 3)
\end{lstlisting}
\end{frame}

\begin{frame}[fragile]
  \frametitle{Hutton's Backup Razor}
\begin{lstlisting}
data MultLang
  = MultLangIntLit Integer
  | Mult MultLang MultLang
  deriving ( Show, Eq )

interpMultLang :: MultLang -> Integer
interpMultLang = \case
  MultLangIntLit i -> i
  Mult l r -> interpMultLang l * interpMultLang r

multLangExp :: MultLang
multLangExp = Mult (MultLangIntLit 1) (MultLangIntLit 3)
\end{lstlisting}
\end{frame}

\begin{frame}[fragile]
  \frametitle{Hutton's Travel Kit}
\begin{lstlisting}
data RingLang
  = RingLangIntLit Integer
  | RingAdd  RingLang RingLang
  | RingMult RingLang RingLang
  deriving ( Show, Eq )

interpRingLang :: RingLang -> Integer
interpRingLang = \case
  RingLangIntLit i -> i
  RingAdd  l r -> interpRingLang l + interpRingLang r
  RingMult l r -> interpRingLang l * interpRingLang r

ringLangExp :: RingLang
ringLangExp = RingMult (RingAdd (RingLangIntLit 3) 
                                (RingLangIntLit 2)) 
                       (RingLangIntLit 3)
\end{lstlisting}
\end{frame}

\section{What's wrong with this picture?}

\begin{frame}[fragile]
  \frametitle{D. R. Y.}
\begin{lstlisting}
data RingLang
  = RingLangIntLit Integer
  | RingAdd  RingLang RingLang
  | RingMult RingLang RingLang
  deriving ( Show, Eq )

interpRingLang :: RingLang -> Integer
interpRingLang = \case
  RingLangIntLit i -> i
  RingAdd  l r -> interpRingLang l + interpRingLang r
  RingMult l r -> interpRingLang l * interpRingLang r
\end{lstlisting}
\end{frame}

\begin{frame}[fragile]
  \frametitle{D. R. Y.}
\begin{lstlisting}

  = RingLangIntLit Integer
  | RingAdd  RingLang RingLang
  | RingMult RingLang RingLang




  RingLangIntLit i -> i
  RingAdd  l r -> interpRingLang l + interpRingLang r
  RingMult l r -> interpRingLang l * interpRingLang r
\end{lstlisting}
\end{frame}

\begin{frame}[fragile]
  \frametitle{D. R. Y.}
\begin{lstlisting}


  | RingAdd  RingLang RingLang
  | RingMult RingLang RingLang





  RingAdd  l r -> interpRingLang l + interpRingLang r
  RingMult l r -> interpRingLang l * interpRingLang r
\end{lstlisting}
\end{frame}

\plain{Nooooooooooooooooooooo- oooooooooooooooooo!}

\begin{frame}
  \frametitle{Let's use category theory!}
  \begin{displaymath}
    \xymatrix{
      \mathtt{AddLang} & \bullet \ar[rdd]_{???} & & \ar[ldd]^{???} \bullet & \mathtt{MultLang} \\
      & & & & \\
      & & \bullet & & \\
      & & \mathtt{RingLang} & & \\
    }
  \end{displaymath}
\end{frame}

\section{Once more, with feeling!}

\begin{frame}[fragile]
  \frametitle{Routine recursion scheme surgery}
\begin{lstlisting}[p]
newtype Fix f = Fix { unFix :: f (Fix f) }

fixFold :: Functor f => (f a -> a) -> (Fix f -> a)
fixFold phi = go where go = phi . fmap go . unFix
\end{lstlisting}
\end{frame}

\begin{frame}[fragile]
  \frametitle{Routine recursion scheme surgery}
\begin{lstlisting}
{-# LANGUAGE UndecidableInstances #-}

deriving instance Show (f (Fix f)) => Show (Fix f)
deriving instance Eq   (f (Fix f)) => Eq   (Fix f)
\end{lstlisting}
\end{frame}

\begin{frame}[fragile]
  \frametitle{Routine recursion scheme surgery}
\begin{lstlisting}
{-# LANGUAGE DeriveFunctor #-}

data AddF  x = AddI  Integer | AddF  x x deriving ( Show, Eq, Functor )
data MultF x = MultI Integer | MultF x x deriving ( Show, Eq, Functor )

type AddLang'  = Fix AddF
type MultLang' = Fix MultF
\end{lstlisting}
\end{frame}

\begin{frame}[fragile]
  \frametitle{Routine recursion scheme surgery}
\begin{lstlisting}
addLangExp' :: AddLang'
addLangExp' = Fix (AddF (Fix (AddI 1)) (Fix (AddI 3)))

multLangExp' :: MultLang'
multLangExp' = Fix (MultF (Fix (MultI 1)) (Fix (MultI 3)))
\end{lstlisting}
\end{frame}

\begin{frame}[fragile]
  \frametitle{Routine recursion scheme surgery}
\begin{lstlisting}
-- Smart constructors! Yessssssss!
addI :: Integer -> AddLang'
addI = Fix . AddI

multI :: Integer -> MultLang'
multI = Fix . MultI

addAdd :: AddLang' -> AddLang' -> AddLang'
addAdd l r = Fix (AddF l r)

multMult :: MultLang' -> MultLang' -> MultLang'
multMult l r = Fix (MultF l r)

addLangExp' :: AddLang'
addLangExp' = addAdd (addI 1) (addI 3)

multLangExp' :: MultLang'
multLangExp' = multMult (multI 1) (multI 3)
\end{lstlisting}
\end{frame}

\begin{frame}[fragile]
  \frametitle{Routine recursion scheme surgery}
\begin{lstlisting}
interpAddF :: AddF Integer -> Integer
interpAddF = \case
  AddI i   -> i
  AddF l r -> l + r

interpMultF :: MultF Integer -> Integer
interpMultF = \case
  MultI i   -> i
  MultF l r -> l * r

interpAddLang' :: AddLang' -> Integer
interpAddLang' = fixFold interpAddF

interpMultLang' :: MultLang' -> Integer
interpMultLang' = fixFold interpMultF
\end{lstlisting}
\end{frame}

\begin{frame}[fragile]
  \frametitle{Routine recursion scheme surgery}
\begin{lstlisting}
{-# LANGUAGE TypeOperators #-}

data (f :+: g) x 
  = Inl (f x) 
  | Inr (g x) 
  deriving ( Eq, Show, Functor )

-- Natural :+: eliminator
foldSum :: (f a -> a) -> (g a -> a) -> ((f :+: g) a -> a)
foldSum falg galg = \case
  Inl fa -> falg fa
  Inr ga -> galg ga
\end{lstlisting}
\end{frame}

\begin{frame}[fragile]
  \frametitle{The prize!}
\begin{lstlisting}
type RingLang' = Fix (AddF :+: MultF)

-- More smart constructors!
ringI :: Integer -> AddLang'
ringI = Fix . Inl . AddI  -- why not via MultI?

ringAdd :: AddLang' -> AddLang' -> AddLang'
ringAdd l r = Fix (Inl (AddF l r))

ringMult :: MultLang' -> MultLang' -> MultLang'
ringMult l r = Fix (Inr (MultF l r))

ringLangExp' :: RingLang'
ringLangExp' = ringMult (ringAdd (ringI 3) (ringI 2)) (ringI 3)

interpRingLang' :: RingLang' -> Integer
interpRingLang' = fixFold (foldSum interpAddF interpMultF)
\end{lstlisting}
\end{frame}

\begin{frame}[fragile]
\begin{lstlisting}
Prelude> interpRingLang' ringLangExp'

\end{lstlisting}
\end{frame}

\begin{frame}[fragile]
\begin{lstlisting}
Prelude> interpRingLang' ringLangExp'
15
\end{lstlisting}
\end{frame}

\plain{Yessssssss! HASKELL! \\ \textbackslash{}textbackslash o/}

\begin{frame}
  \frametitle{D. R. Y. Stats}
  \begin{center}
    40 additions, 3 new PRAGMAs
    
    \pause

    Used \texttt{Fix} in anger
  \end{center}
\end{frame}

\plain{Yessssssss! HASKELL! \\ (+1 Functional Programming)}

\begin{frame}
  \frametitle{In all seriousness...}
  Generalizing \lstinline{:+:}

  Wouter Swierstra, \textit{Data types \`a la carte}.

  Really cool.
\end{frame}

\section{Data types \`a la Carette, Kiselyov, and Shan}

\begin{frame}
  Let's try this again...
\end{frame}

\begin{frame}[fragile]
  \frametitle{Design by wishful thinking}
\begin{lstlisting}
class Adds v where
  add :: v -> v -> v

class Multiplies v where
  mult :: v -> v -> v
\end{lstlisting}
\end{frame}

\begin{frame}[fragile]
\begin{lstlisting}
instance Adds Integer where add = (+)
instance Multiplies Integer where mult = (*)

addsExp :: Integer
addsExp = add 1 3

multsExp :: Integer
multsExp = mult 1 3
\end{lstlisting}
\end{frame}

\begin{frame}[fragile]
\begin{lstlisting}
class FromInteger v where
  i :: Integer -> v

instance FromInteger Integer where
  i = id
\end{lstlisting}
\end{frame}

\begin{frame}[fragile]
\begin{lstlisting}
addsExp :: (FromInteger v, Adds v) => v
addsExp = add (i 1) (i 3)

multsExp :: (FromInteger v, Multiplies v) => v
multsExp = mult (i 1) (i 3)
\end{lstlisting}
\end{frame}

\begin{frame}[fragile]
\begin{lstlisting}
{-# LANGUAGE ConstraintKinds #-}

type Rings v = (FromInteger v, Adds v, Multiplies v)

ringsExp :: Rings v => v
ringsExp = mult (add (i 3) (i 2)) (i 3)
\end{lstlisting}
\end{frame}

\begin{frame}[fragile]
\begin{lstlisting}
Prelude> ringsExp

\end{lstlisting}
\end{frame}

\begin{frame}[fragile]
\begin{lstlisting}
Prelude> ringsExp
15
\end{lstlisting}
\end{frame}

\begin{frame}[fragile]
\begin{lstlisting}
Prelude> ringsExp :: Integer
15
\end{lstlisting}
\end{frame}

\begin{frame}[fragile]
  \frametitle{But then!}
\begin{lstlisting}
instance FromInteger RingLang where i    = RingLangIntLit
instance Adds        RingLang where add  = RingAdd
instance Multiplies  RingLang where mult = RingMult
\end{lstlisting}
\end{frame}

\begin{frame}[fragile]
\begin{lstlisting}
Prelude> ringsExp :: RingLang

\end{lstlisting}
\end{frame}

\begin{frame}[fragile]
\begin{lstlisting}
Prelude> ringsExp :: RingLang
RingMult (RingAdd (RingLangIntLit 3) 
                  (RingLangIntLit 2)) 
         (RingLangIntLit 3)
\end{lstlisting}
\end{frame}

\begin{frame}
  \frametitle{Recap}
  \begin{itemize}
  \pause
  \item One-line composabiity
  \pause
  \item Needn't mention concrete data types or \lstinline{Inl}/\lstinline{Inr}
    \pause
    \begin{itemize}
    \item ``tagless''
    \end{itemize}
  \pause
  \item But if we have them, we can still recover ``raw'' ASTs
  \end{itemize}
\end{frame}

\section{"Does is scale?"}

\begin{frame}
  Yes.
\end{frame}

\section{The "Monad Transformer Library"}

\begin{frame}
  Or: ``Haskell2010 Composable Effects Through Prolog Technology''
\end{frame}

\begin{frame}
  Or: ``Haskell2010 Composable Effects Through Typeclass Technology''
\end{frame}

\begin{frame}[fragile]
\begin{lstlisting}
import Control.Monad.State

inc :: MonadState Int m => m Int
inc = do
  count <- get
  set (count + 1)
  return count
\end{lstlisting}
\end{frame}

\begin{frame}[fragile]
\begin{lstlisting}
-- let's get a concrete stack from `transformers`
import Control.Monad.Trans.State (runState)

Prelude> flip runState 0 inc 
(1, 1)
\end{lstlisting}
\end{frame}

\begin{frame}[fragile]
\begin{lstlisting}
-- let's get a concrete stack from `transformers`
import Control.Monad.Trans.State (runState)

Prelude> flip runState 0 (inc :: StateT Int Identity Int)
(1, 1)
\end{lstlisting}
\end{frame}

\begin{frame}
  Solve your monad stacks at compile time! \pause Not runtime\pause, nor write-time.
\end{frame}

\section{Complaints}

\begin{frame}[fragile]
  \frametitle{Just a few points in the lattice}
\begin{lstlisting}
MonadTrans (ContT r) Source	 
Monad (ContT r m) Source	 
Functor (ContT r m) Source	 
Applicative (ContT r m) Source	 
MonadIO m => MonadIO (ContT r m) Source	 
\end{lstlisting}
\end{frame}

\begin{frame}[fragile]
  \frametitle{All points in the lattice}
\begin{lstlisting}
MonadCont (ContT r m) Source	 
\end{lstlisting}
  \pause
\begin{lstlisting}
MonadCont m => MonadCont (MaybeT m) Source	 
MonadCont m => MonadCont (ListT m) Source	 
MonadCont m => MonadCont (IdentityT m) Source	 
(Monoid w, MonadCont m) => MonadCont (WriterT w m) Source	 
(Monoid w, MonadCont m) => MonadCont (WriterT w m) Source	 
(Error e, MonadCont m) => MonadCont (ErrorT e m) Source	 
MonadCont m => MonadCont (ExceptT e m) Source	 
MonadCont m => MonadCont (StateT s m) Source	 
MonadCont m => MonadCont (StateT s m) Source	 
MonadCont m => MonadCont (ReaderT r m) Source	 
(Monoid w, MonadCont m) => MonadCont (RWST r w s m) Source	 
(Monoid w, MonadCont m) => MonadCont (RWST r w s m) Source
...
-- This gets quite complex
\end{lstlisting}
\end{frame}

\begin{frame}
  \frametitle{All points in the lattice}
  But this is just the name of the game! Effects don't commute.
\end{frame}

\begin{frame}[fragile,fragile]
  \frametitle{Lose control of ordering}
\begin{lstlisting}
op :: (MonadError e m, MonadState s m) => m ()
\end{lstlisting}
  \pause
\begin{lstlisting}
op ==> StateT s (Either e) ()
op ==> EitherT e (State s) ()
\end{lstlisting}
  \pause
No right answer!
\end{frame}

\begin{frame}
  \frametitle{Loose control of ordering}
  This is why we have laws.
  \pause
  Combining laws ``correctly'' \textit{is} hard.
\end{frame}

\begin{frame}[fragile]
  \frametitle{Loose control of ordering}
\begin{lstlisting}
class (MonadState s m, MonadError e m) => MonadParser s e m where {}
\end{lstlisting}
\end{frame}

\begin{frame}[fragile]
  \frametitle{Tight denotation of semantics}
\begin{lstlisting}
-- Tighten to your domain
class MonadParser m where
  type family Char m :: *
  type family Error m :: *

  peekChar :: m (Char m)
  getChar  :: m (Char m)
  
  failParse :: Error m -> m a
\end{lstlisting}
\end{frame}

\section{Thanks!}

\begin{frame}
  \frametitle{Tweet at me!}
  \begin{center}
    @sdbo
  \end{center}
\end{frame}

\end{document}

%%% Local Variables: 
%%% coding: utf-8
%%% mode: latex
%%% TeX-engine: xetex
%%% End: 