\documentclass[pdf]{beamer}
\usepackage{amsmath}
\usepackage{graphicx}
\usepackage{hyperref}
\usepackage{listings}
\usepackage{tcolorbox}
\usepackage[all]{xy}

\mode<presentation>{}

% ------------------------------------------------------------------------------
% Theme

\usetheme[usetitleprogressbar, nosmallcapitals, usetotalslideindicator]{m}

\lstloadlanguages{Haskell}
\lstnewenvironment{code}
    {\lstset{}%
      \csname lst@SetFirstLabel\endcsname}
    {\csname lst@SaveFirstLabel\endcsname}
    \lstset{
      basicstyle=\ttfamily\footnotesize,
      flexiblecolumns=false,
      basewidth={0.5em,0.45em},
      literate={+}{{$+$}}1 {/}{{$/$}}1 {*}{{$*$}}1
               {\\\\}{{\char`\\\char`\\}}1
               {=>}{{$\Rightarrow$}}2
               {forall}{{$\forall$}}2
               {->}{{$\rightarrow$}}2
               {<-}{{$\leftarrow$}}2
               {>>}{{>>}}3 {>>=}{{>>=}}3,
      commentstyle={\ttfamily\color{gray}},
      language=haskell
    }
    
% ------------------------------------------------------------------------------
% Presentation

\title{Give me freedom!}
\subtitle{Or let me forget}
\date{\today}
\author{Joseph Tel Abrahamson / @sdbo / \texttt{github.com/tel} }

\renewcommand{\to}{\ensuremath{\rightarrow}}
\DeclareMathOperator{\Free}{\texttt{Free}}
\DeclareMathOperator{\Forget}{\texttt{Forget}}
\DeclareMathOperator{\Monad}{\texttt{Monad}}
\DeclareMathOperator{\Functor}{\texttt{Functor}}
\DeclareMathOperator{\ty}{\texttt{ :: }}

\begin{document}


\maketitle

\begin{frame}
  \frametitle{Synopsis}
  \begin{itemize}
  \item Ways of seeing Freedom: a noun, an adjective, a verb
  \item Thinking through Freedom as a process
  \item Using Category Theory as a tool of insight
  \end{itemize}
\end{frame}

\section{A first glimpse of Freedom}

\begin{frame}[fragile]
  \frametitle{\texttt{Free} is a noun}
  \pause
\begin{lstlisting}
data Free f a
  = Return a
  | Free (f (Free f a))
\end{lstlisting}
\end{frame}

\begin{frame}[fragile,fragile]
  \frametitle{What are Free Monads anyway?}
  \pause
\begin{lstlisting}
newtype Identity a = Identity a
newtype Fix f = Fix (f (Fix f))

-- Free f a ~= Identity a + Fix f
\end{lstlisting}
  \pause
\begin{lstlisting}
-- I don't know, something like that, not quite kind of?
\end{lstlisting}
\end{frame}

\plain{What's going on here?}

\begin{frame}[fragile]
\begin{lstlisting}
lift     :: Functor f => f a -> Free f a
foldFree :: Monad m => (forall x . f x -> m x) -> (Free f a -> m a)
\end{lstlisting}
\end{frame}

\begin{frame}[fragile]
  \frametitle{Free Monads as "interpreters"}
\begin{lstlisting}
data TeletypeF a 
  = PutStrLn String a 
  | GetLine (String -> a)
    deriving ( Functor )
                     
type Teletype = Free TeletypeF

putStrLnTT :: String -> Teletype ()
putStrLnTT line = lift (PutStrLn line ())

getLineTT :: Teletype String
getLineTT = lift (GetLine id)
\end{lstlisting}
\end{frame}

\begin{frame}[fragile]
  \frametitle{Very nice embedded DSLs... for \textit{less}!}
\begin{lstlisting}
echoTT :: Teletype ()
echoTT = forever $ do
  line <- getLineTT
  putStrLineTT line
\end{lstlisting}
\end{frame}

\begin{frame}[fragile]
  \frametitle{Very nice embedded DSLs... for \textit{less}!}
\begin{lstlisting}
interp :: TeletypeF a -> IO a
interp x = case x of
  PutStrLn line a -> putStrLn line >> return a
  GetLine next -> do
    line <- getLine
    return (next line)

echoIO :: IO ()
echoIO = fold interp echoTT
\end{lstlisting}
\end{frame}

\begin{frame}
  ``Less'' $\neq$ Free
\end{frame}

\plain{What's going on here?}

\section{Free is an adjective}

\begin{frame}
  \frametitle{"Free" things!}
  \begin{itemize}
  \item \lstinline{Free} makes free \lstinline{Monad}s! \pause
  \item Free \lstinline{Monoid}s are lists?
  \end{itemize}
\end{frame}

\begin{frame}[fragile]
  \frametitle{Free \texttt{Monoid}s are lists}
\begin{lstlisting}
pure    :: a -> [a]
foldMap :: Monoid m => (a -> m) -> ([a] -> m)
\end{lstlisting}
  \pause  
  ``\lstinline{Foldable} just means \lstinline{toList}''
\end{frame}

\begin{frame}[fragile]
  \frametitle{Free \texttt{Monoid}s are lists}
\begin{lstlisting}
lift     :: Functor f => f a -> Free f a
foldFree :: Monad m => (forall x . f x -> m x) -> (Free f a -> m a)
\end{lstlisting}
\end{frame}

\begin{frame}
  \frametitle{"Free" things!}
  \begin{itemize}
  \item \lstinline{Free} makes free \lstinline{Monad}s! \pause
  \item Free \lstinline{Monoid}s are lists! \pause
  \item We can make free \lstinline{Applicative}s, I hear \pause
  \item Can there be free things of any kind? Sure looks like it! \pause
  \item Let's free all the things!
  \end{itemize}
\end{frame}

\begin{frame}
  \begin{itemize}
  \item Lists are the ``largest'' \lstinline{Monoid}s
  \item Lists are the ``simplest'' \lstinline{Monoid}s \pause
  \item What are the largest and simplest examples of other things?
  \end{itemize}
\end{frame}

\begin{frame}
  \frametitle{More questions than answers}
  \begin{itemize}
  \item \lstinline{Free f} is the ``largest'' \lstinline{Monad}?  \pause
  \item Does that mean that both \lstinline{Free TeletypeF} and
    \lstinline{Free []} are \textit{both} the ``largest'' \lstinline{Monad}?
    \pause
  \item I hear that \lstinline{Free f} and \lstinline{Operational f} are
    \textit{both} free monads? \pause But they're not isomorphic.
  \end{itemize}
\end{frame}

\plain{Wat.}
\plain{What's going on here?}

\section{Freedom is a process}

\begin{frame}[fragile]
  \frametitle{What is \texttt{Free}, really?}
  
  \uncover<3->{But really more like...}

  \begin{align*}
    & \action<+->{\mathtt{>}\ \mathtt{:kind}\ \Free} \\
    & \action<+->{
        \Free \ty 
          (\star \to \star)_{\uncover<3->{\Functor}} \to
          (\star \to \star)_{\uncover<3->{\Monad}}
      }
  \end{align*}
  
  \pause

  \begin{tcolorbox}[boxrule=0pt, arc=0pt, outer arc=0pt]
\begin{lstlisting}
-- remember...
instance Functor f => Monad (Free f)
\end{lstlisting}
  \end{tcolorbox}
\end{frame}

\plain{Oh, an \textit{arrow}! \pause Time to use some category theory!}

\begin{frame}
  \frametitle{A picture of ``Free monads"}
  \begin{description}
  \item[$\Free_{\Monad}$] 
    \begin{displaymath}
      \mathtt{Functor}\xymatrix{\bullet \ar[r]^{\Free} & \bullet}\mathtt{Monad}
    \end{displaymath}
  \pause
  \item[$\mathtt{List}$]
    \begin{displaymath}
      \mathtt{Hask}\xymatrix{\bullet \ar[r]^{\Free} & \bullet}\mathtt{Monoid}
    \end{displaymath}
  \pause
  \item[$\mathtt{Coyoneda}$] 
    \begin{displaymath}
      \mathtt{Hask}_{(\star\to\star)} \xymatrix{\bullet \ar[r]^{\Free} & \bullet}\mathtt{Functor}
    \end{displaymath}
  \end{description}
\end{frame}

\begin{frame}
  \frametitle{Dualize!}
  \begin{center}
    \begin{displaymath}
      \mathtt{Functor}
      \xymatrix{
        \bullet \ar@/^/[r]^{\Free} & 
        \bullet
      }
      \mathtt{Monad}
    \end{displaymath}
  \end{center}
\end{frame}

\begin{frame}
  \frametitle{Dualize!}
  \begin{center}
    \begin{displaymath}
      \mathtt{Functor}
      \xymatrix{
        \bullet \ar@/^/[r]^{\Free} & 
        \bullet \ar@/^/[l]^{\Forget}
      }
      \mathtt{Monad}
    \end{displaymath}
  \end{center}
\end{frame}

\begin{frame}
  \begin{itemize}
  \item If $\Forget \ty \Monad \to \Functor$ \textit{forgets} that some type is
    a \lstinline{Monad}... \pause
  \item Is $\Free \ty \Functor \to \Monad$ remembering it?
  \end{itemize}
\end{frame}

\begin{frame}
  \frametitle{Not quite}
  \begin{align*}
    (\Free \circ \Forget)(M) &\neq M \\
    (\Forget \circ \Free)(F) &\neq F
  \end{align*}
  \pause
  For good reason!
\end{frame}

\begin{frame}
  \frametitle{Just right}
  If
  \begin{flalign*}
    M = \Free(F)
  \end{flalign*}
  for some \texttt{Functor} $F$, then
  \begin{flalign*}
    (\Free \circ \Forget)(M) = M
  \end{flalign*}
\end{frame}

\begin{frame}
  \frametitle{Just right}
  If
  \begin{flalign*}
    F = \Forget(M)
  \end{flalign*}
  for some \texttt{Monad} $M$, then
  \begin{flalign*}
    (\Forget \circ \Free)(F) = F
  \end{flalign*}
\end{frame}

\begin{frame}
  \frametitle{Adjunctions}
  \begin{align*}
    Free &\dashv Forget
  \end{align*}
  \begin{align*}
    Free \circ Forget \circ Free &= Free \\
    Forget \circ Free \circ Forget &= Forget 
  \end{align*}
\end{frame}

\begin{frame}[fragile]
  \frametitle{What does an Adjunction buy us?}
  \begin{align*}
    F &: \mathcal{C} \to \mathcal{D} \\
    G &: \mathcal{D} \to \mathcal{C}
  \end{align*}
  \pause
  \begin{align*}
    \forall c : \mathcal{C}, d &: \mathcal{D}, \mathcal{D}(F c, d) \equiv \mathcal{D}(c, G d)
  \end{align*}
\end{frame}

\begin{frame}[fragile]
  \frametitle{What does an Adjunction buy us?}
\begin{lstlisting}
type Forget f a = f a

-- "Natural transformations"
type f :-> g = forall x . f x -> g x

fwd :: Monad m => (Free f :-> m) -> (f :-> Forget m)
bwd :: Monad m => (f :-> Forget m) -> (Free f :-> m)
\end{lstlisting}
\end{frame}

\begin{frame}[fragile]
  \frametitle{What does an Adjunction buy us?}
\begin{lstlisting}





fwd :: Monad m => (forall x . Free f x -> m x) -> (f a -> m a)
bwd :: Monad m => (forall x . f x -> m x) -> (Free f a -> m a)
\end{lstlisting}
\end{frame}

\begin{frame}[fragile]
\begin{lstlisting}
foldFree :: Monad m => (forall x . f x -> m x) -> (Free f a -> m a)
foldFree = bwd
\end{lstlisting}
\end{frame}

\begin{frame}[fragile]
\begin{lstlisting}
idFree :: Free f x -> Free f x
idFree = id

-- m ~ Free f
lift :: f a -> Free f a
lift = fwd idFre
\end{lstlisting}
\end{frame}

\begin{frame}
  \frametitle{What does an Adjunction buy us?}
  Everything we need.
\end{frame}

\section{Bonus: Freedom for everyone}

\begin{frame}[fragile]
  \frametitle{Definition by elimination}
  \pause
\begin{lstlisting}
curious :: _
curious = flip bwd
\end{lstlisting}
\end{frame}

\begin{frame}[fragile]
  \frametitle{Definition by elimination}
\begin{lstlisting}
curious :: Monad m => Free f a -> (f :-> m) -> m a
curious = flip bwd
\end{lstlisting}
\end{frame}

\begin{frame}[fragile]
  \frametitle{Definition by elimination}
\begin{lstlisting}


newtype Free f a 
  = Free { runFree :: forall m . Monad m => (f :-> m) -> m a }
\end{lstlisting}
\end{frame}

\begin{frame}[fragile]
  \frametitle{Definition by elimination}
\begin{lstlisting}
{-# LANGUAGE ConstraintKinds #-}

newtype Free c f a 
  = Free { runFree :: forall m . c m => (f :-> m) -> m a }
\end{lstlisting}
\end{frame}

\begin{frame}[fragile]
  \frametitle{Definition by elimination}
\begin{lstlisting}
{-# LANGUAGE ConstraintKinds #-}

newtype HFree c f a 
  = HFree { runHFree :: forall m . c m => (f :-> m) -> m a }
\end{lstlisting}
\end{frame}

\begin{frame}[fragile]
  \frametitle{Definition by elimination}
\begin{lstlisting}
{-# LANGUAGE ConstraintKinds #-}

newtype Free c a 
  = Free { runFree :: forall r . c r => (a -> r) -> r }
\end{lstlisting}
\end{frame}

\begin{frame}[fragile]
  \frametitle{Definition by elimination}
\begin{lstlisting}
free :: [a] -> Free Monoid a
free as = Free $ \ar -> foldMap ar as

unfree :: Free Monoid a -> [a]
unfree f = runFree f (\x -> [x])
\end{lstlisting}
\end{frame}

\section{Thanks!}

\begin{frame}
  \frametitle{Tweet at me!}
  \begin{center}
    @sdbo
  \end{center}
\end{frame}

\end{document}

%%% Local Variables: 
%%% coding: utf-8
%%% mode: latex
%%% TeX-engine: xetex
%%% End: 