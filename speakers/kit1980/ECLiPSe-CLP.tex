\documentclass{beamer}
\usetheme{Boadilla}

\usepackage{hyperref}

\title{Introduction to constraint logic programming with ECLiPSe}
\author{Sergii Dymchenko}
\institute{\url{http://sdymchenko.com}}
\date{May 22, 2015}

\begin{document}

\begin{frame}
\titlepage
\end{frame}

\begin{frame}
\frametitle{CLP}
Constraint Logic Programming (CLP) is an augmentation of logic programming paradigm where relations between variables are specified with constraints.
\begin{itemize}
\item $X \le 10$
\item $X > Y$
\item $Y = 9$
\end{itemize}
\end{frame}

\begin{frame}
\frametitle{CLP: Pros}
\begin{itemize}
\item Declarative
\item Compact
\item Understandable
\item Easy to modify
\item Often fast enough
\end{itemize}
\end{frame}

\begin{frame}
\frametitle{CLP: Cons}
\begin{itemize}
\item Running time extremely depends on the instance
\item Running time extremely depends on heuristics
\item Optimization (vs any feasible solution) is slow
\item Based on non-mainstream logic programming paradigm
\end{itemize}
\end{frame}

\begin{frame}
\frametitle{ECLiPSe}
ECLiPSe CLP (\url{http://eclipseclp.org/}) is an open-source Prolog-based system which
aims to serve as a platform for integrating various logic programming extensions, in particular constraint logic programming.
\end{frame}

\begin{frame}
\frametitle{ECLiPSe: libraries}
\begin{itemize}
\item \texttt{ic}: interval arithmetic constraint library
\item \texttt{gfd}: interface to Gecode (\url{http://www.gecode.org/})
\item Other CLP libraries (constraints on graphs, sets, \dots)
\item Interfaces to linear programming solvers
\item A lot of them: \url{http://eclipseclp.org/doc/bips/index.html}
\end{itemize}
\end{frame}

\begin{frame}
\frametitle{Demo some Prolog in ECLiPSe}
Demo some Prolog in ECLiPSe (incl. \texttt{append}).
\end{frame}

\begin{frame}
\frametitle{TPK}
TPK is a simple algorithm proposed by D.\,E.\,Knuth and L.\,T.\,Pardo in ``The Early Development
of Programming Languages''. It is used to demonstrate some basic syntax of a language beyond the ``Hello, World!''.
\begin{itemize}
\item Prompt for 11 real numbers ($a_0 \dots a_{10}$)
\item For each $a_i$ compute $b_i = f(a_i)$, where $f(t) = \sqrt{|t|} + 5 {t}^3$
\item For $i = 10 \dots 0$ (in that order) output a pair $(i, b_i)$ if $b_i \leq 400$, or \\$(i,$ TOO LARGE$)$ otherwise
\end{itemize}
Demo \texttt{tpk.ecl}.
\end{frame}

\begin{frame}
\frametitle{Arithmetic in Prolog}
\begin{itemize}
\item \texttt{is}-based arithmetic in Prolog is not relational
\item CLP can be seen as extension that brings relational arithmetic to Prolog
\item Demo $c = a / b$
\end{itemize}
\end{frame}


\begin{frame}
\frametitle{Exercise: relational factorial}
\begin{itemize}
\item Complete code in \texttt{factorial.ecl}
\item \texttt{factorial} should work ``both ways'': each argument can be input or output
\item Demo \texttt{factorial.ecl}
\end{itemize}
\end{frame}

\begin{frame}
\frametitle{SEND + MORE = MONEY}
\begin{itemize}
\item Cryptarithmetic puzzle (\url{http://en.wikipedia.org/wiki/Verbal_arithmetic})
\item Demo \texttt{money.ecl}
\end{itemize}
\end{frame}

\begin{frame}
\frametitle{Constraint propagation and search}
\begin{itemize}
\item Just constraint propagation is not enough in general case
\item Search is needed
\item Parameters: \url{http://eclipseclp.org/doc/bips/lib/ic/search-6.html}
\item Demo \texttt{alldifferent.ecl}
\end{itemize}
\end{frame}

\begin{frame}
\frametitle{N-queens puzzle}
\begin{itemize}
\item Place N non-attacking chess queens on an $N \times N$ board
\item \url{http://en.wikipedia.org/wiki/Eight_queens_puzzle}
\item Different formulations are possible: $N^2$ 0/1 vars, $2 \times N$ 1..N vars, $N$ 1..N vars
\item Demo \texttt{queens.ecl}
\end{itemize}
\end{frame}

\begin{frame}
\frametitle{Exercise: playing with N-queens}
Try to introduce different changes:
\begin{itemize}
\item Different variable ordering heuristic
\item Different value ordering heuristic
\item Arrays instead of lists
\item \texttt{gfd} instead of \texttt{ic}
\item Maybe something else
\end{itemize}
Observe changes in running time for larger problem instances.
\end{frame}

\begin{frame}
\frametitle{Optimization}
\begin{itemize}
\item Optimal vs any feasible solution
\item branch-and-bound: \url{http://eclipseclp.org/doc/bips/lib/branch_and_bound/index.html}
\item Demo \texttt{alldifferent-bb.ecl}
\end{itemize}
\end{frame}

\begin{frame}
\frametitle{Real world}
\begin{itemize}
\item \url{http://eclipseclp.org/reports/index.html\#Applications}
\item Almost any combinatorial problem
\item Interfaces to Java, C++, Python, \dots
\item Development tools: testing, debugging, profiling, \dots
\end{itemize}
\end{frame}

\begin{frame}
\frametitle{More info}
\begin{itemize}
\item ECLiPSe official site -- \url{http://www.eclipseclp.org/}
\item Book ``Constraint Logic Programming using ECLiPSe'' by Krzysztof Apt and Mark Wallace
\item ELearning course by Helmut Simonis -- \url{http://4c.ucc.ie/~hsimonis/ELearning/index.htm}
\item Examples by Hakan Kjellerstrand -- \url{http://www.hakank.org/eclipse/}
\end{itemize}
\end{frame}

\end{document}
